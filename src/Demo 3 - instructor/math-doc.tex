\documentclass{article}
\usepackage{amsmath}

\newcommand{\flow}{\rightleftharpoons}

\begin{document}
\title{Typesetting Mathematics in \LaTeX}
\author{Nauman \\ recluze@gmail.com}
\maketitle


\section{Introduction} 
\LaTeX\ is extremely powerful when it comes to typesetting mathematics. It's one of the core strengths of this system. 

\section{Displaying Mathematics}
There are two ways of displaying maths. One is \emph{inline} and the other is \emph{display} format -- in which the whole math sits on its own set of lines.


\subsection{Inline Mode}
We are going to insert a mathematics equation inline here using a pair of \$ signs:  $E=mc^2$    . As you can see, the display (such as line spacing) does not get messed up by the mathematics as it does with word processing softwares. 

\subsection{Display Mode}
We can also display equations in their own set of lines. To do this, we can use the equation environment. 

\begin{equation}\label{eq:emc}
E=mc^2
\end{equation}

As you can see, \LaTeX\ inserts the equation number automatically. We can refer to it using the \verb|\ref| command just as we referred to sections, figures and tables. (E.g. Equation~\ref{eq:emc}.) To get rid of the equation number, simply use the \emph{star variant} of the equation environment. (For this, you need the \texttt{amsmath} package.)

\begin{equation*}
E=mc^2
\end{equation*}

Alternatively, we can use the shorthand keys \verb|\[| and \verb|\]|

\[
E=mc^2
\]

\section{Mathematical Features}
\LaTeX\ has many builtin features and you can get many more easily. Here, we'll see some of these features: 

Addition, subtraction, multiplication and division: 

\[
x+2 - 25 \times 35 \div 98 
\]

Superscripts and subscripts: 

\[ x^2  \]
\[ x_{(i)} \]


Summation, union, intersection, big-union, integral: 

\[ \sum_{i=1}^{n}{i^2} \]
\[ x \cup y \cap z \]
\[ \bigcup_{i=1}^{n}{x_i} \]
\[ \int_0^n{x^2} \]

Fractions, brackets, square root: 

\[ \frac{x}{y} \]
\[ \frac{\sum_i{x^2}}{\int_0^n{x^2}} \]
\[ \sqrt{\frac{\sqrt{36}} {x^5}} \]

\[ 2 \times \left( \frac{34}{\frac{124}{356}}    \right)  \]

Greek letters: 

\[
\alpha + \beta + \gamma^* + \Sigma + \Theta + 2_\epsilon 
\]

Matrices and vectors. For this, you need to include the \texttt{amsmath} package and then use the \texttt{bmatrix} or \texttt{pmatrix} environment: 

\[
\begin{pmatrix}
\frac{a}{44} & b \\ 
c & \sqrt{d} 	
\end{pmatrix}
\]

Accents: 

\[ \hat{x} \]
\[ \hat{\imath} \] 
\[ \dot{x} \]

See the \texttt{Math} menu in the IDE for other operations. You can refer to ``Short Math Guide for \LaTeX'' for a lot more examples. 

\section{Using Symbols}
You might come across situations where you need to find new symbols. For this, you can refer to the ``The Comprehensive \LaTeX Symbols List''.  

\[ x \rightleftharpoons  y \]



(Optional) Since this is a long command, we might want to create a shortcut using the \verb|\newcommand| command in the preamble. This also allows us to later change the symbol without having to change the equations. 

\[ x \flow y \]

\end{document}